\documentclass[11pt, a4paper]{article}
\title{ANLY 500-52-2016 - Assignment 4}
\author{Nelson Corrocher}
\date{July 2016}
\begin{document}
\maketitle
\section{Background}
\paragraph{}The General Deterrence Theory (GDT) proposes that the reason for certain individuals to engage in deviant, anti-social or criminal behavior is that they chose such behaviors based on a rational cost-benefit function. Assuming that individuals make rational decisions, were it possible to increase the costs and risks of pursuing such behavior, these behaviors could be controlled.
\paragraph{}Proving or disproving GDT is difficult due many subjective assumptions and variables involved. For example, each individual will have a different weight for costs and risks, not all made decisions are rational, it is possible to isolate the factors that increase or decrease costs and risks, etc.

\section{Data Points and Observations}  
\paragraph{}First, some way to measure the observed behavior, in this case, deviant, anti-social or criminal, would be needed. In this analysis, this measure would be the dependent variable. Correctly measuring this behavior could be difficulty so using a proxy could be an alternative. An easily collected variable would be criminality rate index, which could be captured collecting information from, for example, police departments.
\paragraph{}Next, some way to measure the costs of unwanted behavior. This is also particularly difficult gather as it is subjective, requires assumptions and, in many cases, can’t be observed.  Again, as a proxy an index based on what is generally believed to be crime deterrents will be used. Some factors could include the size of authority forces that maintain order, the severity of punishment, general tolerance for criminality (corruption, for example), general availability of basic goods and services, etc. That index would be the independent variable. 
\paragraph{}After designing these measures, the next step is to find a major change in the independent variable. For example, a new law that doubled the jailing time for all crimes was approved and came into play in 2011. It should be reasonable to assume that such law would increase the “crime cost index”. Data points would be collected for both the dependent and independent variables monthly retroactively from years before 2011 and until some years after 2011 to compare how the new law affected criminality.

\section{Analysis}
\paragraph{}The collected data would be used to relate the impact in the “crime cost index” and criminality index. Many methods could be used here, but two simple ones that could provide good results would be simple correlation or linear regression. Both could provide a measure of the relation between the two variables (note: relation is normally not causality; in this case, the date of the new law can be used as one of the requirements for causality).
\paragraph{}This approach, however, has many issues:
\begin{itemize}
\item Unless measuring an infinite amount of variables composing the “crime cost index” could be possible, there would never be certainty of the events that were relevant to this analysis. For example, let’s assume the price of many basic products and services went down simultaneously. The criminality could have decreased because the new law or the drop in prices. Even if these two factors were accounted for, a new third factor not considered could had a major effect reducing the criminality.
\item The experiment is almost impossible to replicate for the same reasons stated above. Even if all relevant factors could be correctly accounted for, there could be a different factor affecting the results in the replicated experiment. For example, data collected from two very similar regions with very similar events leaded to different results. It wouldn’t be clear if the differences happened because of mistakes in the original experiment or new unknown but relevant factors existed in the replicated experiment.
\item The indexes/proxies used to measure the variables were imperfect and could fail to capture relevant changes (even if would be possible to isolate the intrinsic noise).
\item No causality (stated in the GDT) has been show as there could be some plausible alternative explanations.
\end{itemize}

\paragraph{}These are just a some of the many problems in proving this theory. Even using other approaches, for example, trying to find one case in which this wouldn’t be valid thus invalidating the GDT, other difficulties would arise (for example, not everyone has the same moral values, some individuals are, in fact, risk-takers and are incited by the risk, etc.).
\section{Conclusion}
\paragraph{}In summary, it’s very difficult, maybe impossible, to prove the general deterrence theory.
\paragraph{}In practice, subsets of GDT (specific behaviors, specific measures, specific group of individuals, etc) are analyzed and through many repeated investigations leading to a similar results are assumed probably true or false.  

\begin{thebibliography}{1}
\bibitem{social2012}Bhattacherjee, A. (2012). Social Science Research: Principles, Methods, and Practices, 2nd edition (2nd ed.).
\bibitem{det2016}Deterrence Theory. (2016). Criminology Wiki. Retrieved 25 July 2016, from http://criminology.wikia.com/wiki/Deterrence\_Theory
\end{thebibliography}
\end{document}










