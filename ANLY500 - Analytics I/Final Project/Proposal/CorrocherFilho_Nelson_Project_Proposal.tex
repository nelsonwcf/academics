\documentclass[11pt, a4paper]{article}
\title{Effects of Financial Statements Releases on Stock of Public Companies}
\author{Adela Devarajan\\Nelson Corrocher\\Nikoloz Lelashvili}
\date{August 2016}
\usepackage{url}
\expandafter\def\expandafter\UrlBreaks\expandafter{\UrlBreaks%  save the current one
	\do\a\do\b\do\c\do\d\do\e\do\f\do\g\do\h\do\i\do\j%
	\do\k\do\l\do\m\do\n\do\o\do\p\do\q\do\r\do\s\do\t%
	\do\u\do\v\do\w\do\x\do\y\do\z\do\A\do\B\do\C\do\D%
	\do\E\do\F\do\G\do\H\do\I\do\J\do\K\do\L\do\M\do\N%
	\do\O\do\P\do\Q\do\R\do\S\do\T\do\U\do\V\do\W\do\X%
	\do\Y\do\Z}
\begin{document}
\maketitle
\section{Hypothesis}
\paragraph{}During an MBA Investment class, there was an idea of using a “straddle” strategy close to the release date of the financial statement from public traded companies. In theory, it shouldn't work better than randomly investing as many market professionals believe the price is already reflected on stocks before the financial statements release to the market. In class simulation, however, that strategy performed really well. Since there is very little evidence, if any, that this strategy does or doesn't perform better than randomly investing in stocks, our proposal is to use data collected from Blue Chip stocks (ones that are negotiated enough to have liquid options available) and compare their price variation close to the release data of Quarterly and Yearly financial to their average price variation. If this happens to be true, it could lead to new opportunities for positive return investments.
\section{Data Collection Methods}
\paragraph{}The data for most stocks is freely available through many sources. For this project, Yahoo Finance has been selected initially due its ability to provide data in directly to csv format. The data for financial statements release dates are freely available from \url{https://www.sec.gov/edgar/searchedgar/companysearch.html}, although not in a use importing friendly format (some manual work is expected here). Bloomberg also seems to provide historical data on SEC fillings releases. The main source for this information will still need to be defined. Finally, we will need information about inflation and risk-free rate, which are readily available from many sites in the web. 
\end{document}











