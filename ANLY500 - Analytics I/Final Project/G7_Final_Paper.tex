\documentclass[english,man]{apa6}

\usepackage{amssymb,amsmath}
\usepackage{ifxetex,ifluatex}
\usepackage{fixltx2e} % provides \textsubscript
\ifnum 0\ifxetex 1\fi\ifluatex 1\fi=0 % if pdftex
  \usepackage[T1]{fontenc}
  \usepackage[utf8]{inputenc}
\else % if luatex or xelatex
  \ifxetex
    \usepackage{mathspec}
    \usepackage{xltxtra,xunicode}
  \else
    \usepackage{fontspec}
  \fi
  \defaultfontfeatures{Mapping=tex-text,Scale=MatchLowercase}
  \newcommand{\euro}{€}
\fi
% use upquote if available, for straight quotes in verbatim environments
\IfFileExists{upquote.sty}{\usepackage{upquote}}{}
% use microtype if available
\IfFileExists{microtype.sty}{\usepackage{microtype}}{}
\usepackage{color}
\usepackage{fancyvrb}
\newcommand{\VerbBar}{|}
\newcommand{\VERB}{\Verb[commandchars=\\\{\}]}
\DefineVerbatimEnvironment{Highlighting}{Verbatim}{commandchars=\\\{\}}
% Add ',fontsize=\small' for more characters per line
\usepackage{framed}
\definecolor{shadecolor}{RGB}{248,248,248}
\newenvironment{Shaded}{\begin{snugshade}}{\end{snugshade}}
\newcommand{\KeywordTok}[1]{\textcolor[rgb]{0.13,0.29,0.53}{\textbf{{#1}}}}
\newcommand{\DataTypeTok}[1]{\textcolor[rgb]{0.13,0.29,0.53}{{#1}}}
\newcommand{\DecValTok}[1]{\textcolor[rgb]{0.00,0.00,0.81}{{#1}}}
\newcommand{\BaseNTok}[1]{\textcolor[rgb]{0.00,0.00,0.81}{{#1}}}
\newcommand{\FloatTok}[1]{\textcolor[rgb]{0.00,0.00,0.81}{{#1}}}
\newcommand{\ConstantTok}[1]{\textcolor[rgb]{0.00,0.00,0.00}{{#1}}}
\newcommand{\CharTok}[1]{\textcolor[rgb]{0.31,0.60,0.02}{{#1}}}
\newcommand{\SpecialCharTok}[1]{\textcolor[rgb]{0.00,0.00,0.00}{{#1}}}
\newcommand{\StringTok}[1]{\textcolor[rgb]{0.31,0.60,0.02}{{#1}}}
\newcommand{\VerbatimStringTok}[1]{\textcolor[rgb]{0.31,0.60,0.02}{{#1}}}
\newcommand{\SpecialStringTok}[1]{\textcolor[rgb]{0.31,0.60,0.02}{{#1}}}
\newcommand{\ImportTok}[1]{{#1}}
\newcommand{\CommentTok}[1]{\textcolor[rgb]{0.56,0.35,0.01}{\textit{{#1}}}}
\newcommand{\DocumentationTok}[1]{\textcolor[rgb]{0.56,0.35,0.01}{\textbf{\textit{{#1}}}}}
\newcommand{\AnnotationTok}[1]{\textcolor[rgb]{0.56,0.35,0.01}{\textbf{\textit{{#1}}}}}
\newcommand{\CommentVarTok}[1]{\textcolor[rgb]{0.56,0.35,0.01}{\textbf{\textit{{#1}}}}}
\newcommand{\OtherTok}[1]{\textcolor[rgb]{0.56,0.35,0.01}{{#1}}}
\newcommand{\FunctionTok}[1]{\textcolor[rgb]{0.00,0.00,0.00}{{#1}}}
\newcommand{\VariableTok}[1]{\textcolor[rgb]{0.00,0.00,0.00}{{#1}}}
\newcommand{\ControlFlowTok}[1]{\textcolor[rgb]{0.13,0.29,0.53}{\textbf{{#1}}}}
\newcommand{\OperatorTok}[1]{\textcolor[rgb]{0.81,0.36,0.00}{\textbf{{#1}}}}
\newcommand{\BuiltInTok}[1]{{#1}}
\newcommand{\ExtensionTok}[1]{{#1}}
\newcommand{\PreprocessorTok}[1]{\textcolor[rgb]{0.56,0.35,0.01}{\textit{{#1}}}}
\newcommand{\AttributeTok}[1]{\textcolor[rgb]{0.77,0.63,0.00}{{#1}}}
\newcommand{\RegionMarkerTok}[1]{{#1}}
\newcommand{\InformationTok}[1]{\textcolor[rgb]{0.56,0.35,0.01}{\textbf{\textit{{#1}}}}}
\newcommand{\WarningTok}[1]{\textcolor[rgb]{0.56,0.35,0.01}{\textbf{\textit{{#1}}}}}
\newcommand{\AlertTok}[1]{\textcolor[rgb]{0.94,0.16,0.16}{{#1}}}
\newcommand{\ErrorTok}[1]{\textcolor[rgb]{0.64,0.00,0.00}{\textbf{{#1}}}}
\newcommand{\NormalTok}[1]{{#1}}

% Table formatting
\usepackage{longtable, booktabs}
\usepackage{lscape}
% \usepackage[counterclockwise]{rotating}   % Landscape page setup for large tables
\usepackage{multirow}		% Table styling
\usepackage{tabularx}		% Control Column width
\usepackage[flushleft]{threeparttable}	% Allows for three part tables with a specified notes section
\usepackage{threeparttablex}            % Lets threeparttable work with longtable

% Create new environments so endfloat can handle them
% \newenvironment{ltable}
%   {\begin{landscape}\begin{center}\begin{threeparttable}}
%   {\end{threeparttable}\end{center}\end{landscape}}

\newenvironment{lltable}
  {\begin{landscape}\begin{center}\begin{ThreePartTable}}
  {\end{ThreePartTable}\end{center}\end{landscape}}

  \usepackage{ifthen} % Only add declarations when endfloat package is loaded
  \ifthenelse{\equal{\string man}{\string man}}{%
   \DeclareDelayedFloatFlavor{ThreePartTable}{table} % Make endfloat play with longtable
   % \DeclareDelayedFloatFlavor{ltable}{table} % Make endfloat play with lscape
   \DeclareDelayedFloatFlavor{lltable}{table} % Make endfloat play with lscape & longtable
  }{}%



% The following enables adjusting longtable caption width to table width
% Solution found at http://golatex.de/longtable-mit-caption-so-breit-wie-die-tabelle-t15767.html
\makeatletter
\newcommand\LastLTentrywidth{1em}
\newlength\longtablewidth
\setlength{\longtablewidth}{1in}
\newcommand\getlongtablewidth{%
 \begingroup
  \ifcsname LT@\roman{LT@tables}\endcsname
  \global\longtablewidth=0pt
  \renewcommand\LT@entry[2]{\global\advance\longtablewidth by ##2\relax\gdef\LastLTentrywidth{##2}}%
  \@nameuse{LT@\roman{LT@tables}}%
  \fi
\endgroup}


  \usepackage{graphicx}
  \makeatletter
  \def\maxwidth{\ifdim\Gin@nat@width>\linewidth\linewidth\else\Gin@nat@width\fi}
  \def\maxheight{\ifdim\Gin@nat@height>\textheight\textheight\else\Gin@nat@height\fi}
  \makeatother
  % Scale images if necessary, so that they will not overflow the page
  % margins by default, and it is still possible to overwrite the defaults
  % using explicit options in \includegraphics[width, height, ...]{}
  \setkeys{Gin}{width=\maxwidth,height=\maxheight,keepaspectratio}
\ifxetex
  \usepackage[setpagesize=false, % page size defined by xetex
              unicode=false, % unicode breaks when used with xetex
              xetex]{hyperref}
\else
  \usepackage[unicode=true]{hyperref}
\fi
\hypersetup{breaklinks=true,
            pdfauthor={},
            pdftitle={Analysis of Stock Price Variance Close To Annual And Quarterly Report Filing Dates},
            colorlinks=true,
            citecolor=blue,
            urlcolor=blue,
            linkcolor=black,
            pdfborder={0 0 0}}
\urlstyle{same}  % don't use monospace font for urls

\setlength{\parindent}{0pt}
%\setlength{\parskip}{0pt plus 0pt minus 0pt}

\setlength{\emergencystretch}{3em}  % prevent overfull lines

\ifxetex
  \usepackage{polyglossia}
  \setmainlanguage{}
\else
  \usepackage[english]{babel}
\fi

% Manuscript styling
\captionsetup{font=singlespacing,justification=justified}
\usepackage{csquotes}
\usepackage{upgreek}



\usepackage{tikz} % Variable definition to generate author note

% fix for \tightlist problem in pandoc 1.14
\providecommand{\tightlist}{%
  \setlength{\itemsep}{0pt}\setlength{\parskip}{0pt}}

% Essential manuscript parts
  \title{Analysis of Stock Price Variance Close To Annual And Quarterly Report
Filing Dates}

  \shorttitle{FINANCIAL REPORTS AND PRICE VARIANCE}


  \author{Nelson Corrocher}

  \def\affdep{{""}}%
  \def\affcity{{""}}%

  \affiliation{
    \vspace{0.5cm}
          \textsuperscript{} Harrisburg University  }

  \authornote{
    \newcounter{author}
    Nelson Corrocher, M.B.A.

                      Correspondence concerning this article should be addressed to Nelson Corrocher. E-mail: \href{mailto:ncorrocher@my.harrisburgu.edu}{\nolinkurl{ncorrocher@my.harrisburgu.edu}}
                }


  \abstract{This paper provides an analysis of the differences between stock price
variation that occurs regularly and the variation that happens close to
10-Q and 10-K filing dates. Using publicly available information,
variation for regular days and for days close to the filing release
dates were calculated and then compared using a same-variance hypothesis
test along with the graphical ploting. The results showed that, in
general, the variance of a stock price close to its filling date is
higher. The implications of this study can lead to a systematic straddle
creation strategy that has potential to provide simplicity of execution
and higher Sharpe Ratio than passive index strategies.}
  \keywords{straddle options financial reports price variance \\

    \indent Word count: 107
  }





\begin{document}

\maketitle

\setcounter{secnumdepth}{0}



During a trading simulation in a graduate investment class at Boston
University in 2014, the winning strategy used an option straddle close
to the release date of the 10-K of a certain public company. This
outcome started a discussion of using a systematic creation of straddles
close to filing dates of companies. In the context of financial
investment, a straddle is a neutral option strategy involving the
simultaneous buying of a put and a call of the same underlying stock,
expecting a great price variation before the option expiration date.
Creating an option straddle close to the filing date of the financial
statement from publicly traded companies has been used by fund managers
to try to improve their funds returns. However, since they use
additional and often undisclosed criteria to decide whose company's
stocks they straddle, there is neither evidence that the variance is
indeed greater close to financial report filing dates, nor that the
strategy, by itself, is profitable. The goal of this research is to
compare the differences in stock price variation that happens on a daily
basis to those that happens close to 10-Q and 10-K filing dates using
publicly available information. For the validation of the assumption
that price variation is indeed higher close to filing dates, five years
of historical stock prices will be collected for the top S\&P index
composing companies, also called Blue Chips (which have greater options
liquidity - used in straddles) along with inflation rates and SEC filing
dates. The stock prices will be adjusted to control for inflation and
their daily price variation will be computed. The filings dates from
each company will be used to mark the days that we expect to see greater
variation. In this reserach we are using day minus 2 to day plus 2 on
the filling dates to compose our test sample. The remaining days will be
used as the control sample. Next, both samples will be first compared
using a same-variance hypothesis test along with the graphical ploting
to show the differences visually, if any. Based on the strategy used by
the fund managers, it is expected that the test sample would show
greater variance and thus, enabling further research into applying this
concept in formulating a strategy and running simulations. Formulating
the strategy and running the simulations are not in this paper's scope
and will be done at a later time.

\section{Methods}\label{methods}

\subsection{Participants}\label{participants}

No direct participants were involved in this research as the main inputs
for the analysis were financial market data that are publicly available.

\subsection{Measures and Procedures}\label{measures-and-procedures}

This entire paper has been constructed almost entirely by using R
scripting (exception for Inflation Daily Rates, explained below). For
replicability purposes, the methodology is going to be explained through
the commented script, enabling the reader to change it for his own
needs. The code below is used to prepare the environment for the script.
\textbf{Note}: the SETWD function in the block above sets the working
directory and should be set to the folder where the inputs and outputs
would be located.

\begin{Shaded}
\begin{Highlighting}[]
\NormalTok{list.of.packages <-}\StringTok{ }\KeywordTok{c}\NormalTok{(}\StringTok{"ggplot2"}\NormalTok{, }\StringTok{"xlsx"}\NormalTok{, }\StringTok{"data.table"}\NormalTok{)}
\NormalTok{for(i in list.of.packages) \{}
  \NormalTok{if(!}\KeywordTok{is.installed}\NormalTok{(i)) \{}
  \KeywordTok{install.packages}\NormalTok{(i)}
  \NormalTok{\}}
\NormalTok{\}}
\KeywordTok{library}\NormalTok{(data.table)}
\KeywordTok{library}\NormalTok{(xlsx)}
\KeywordTok{library}\NormalTok{(ggplot2)}
\end{Highlighting}
\end{Shaded}

In the section below we select the stocks being used in the Analysis
using their tickers. The stocks used are the top comprising blue chips
that are part of the S\&P index. As long as yahoo finance website its
data, any ticker is valid to be added below. To add a new one, put the
ticker and the company name used on the SEC filings (case-sensitive) in
the array below. To get the correct ticker and company filing, visit
\url{https://www.sec.gov/edgar/searchedgar/companysearch.html}.
\textbf{Note}: In some cases, companies change their filing names (for
example, AA - Alcoa). This requires a code modification, also explained
further below.

\begin{Shaded}
\begin{Highlighting}[]
\NormalTok{stocks <-}\StringTok{ }\KeywordTok{data.frame}\NormalTok{(}
  \KeywordTok{c}\NormalTok{(}
    \StringTok{"IBM"}\NormalTok{,}
    \StringTok{"XOM"}\NormalTok{,}
    \StringTok{"CVX"}\NormalTok{,}
    \StringTok{"PG"}\NormalTok{,}
    \StringTok{"MMM"}\NormalTok{,}
    \StringTok{"JNJ"}\NormalTok{,}
    \StringTok{"MCD"}\NormalTok{,}
    \StringTok{"WMT"}\NormalTok{,}
    \StringTok{"UTX"}\NormalTok{,}
    \StringTok{"KO"}\NormalTok{,}
    \StringTok{"BA"}\NormalTok{,}
    \StringTok{"CAT"}\NormalTok{,}
    \StringTok{"JPM"}\NormalTok{,}
    \StringTok{"VZ"}\NormalTok{,}
    \StringTok{"T"}\NormalTok{,}
    \StringTok{"DD"}\NormalTok{,}
    \StringTok{"MRK"}\NormalTok{,}
    \StringTok{"DIS"}\NormalTok{,}
    \StringTok{"HD"}\NormalTok{,}
    \StringTok{"MSFT"}\NormalTok{,}
    \StringTok{"AXP"}\NormalTok{,}
    \StringTok{"BAC"}\NormalTok{,}
    \StringTok{"PFE"}\NormalTok{,}
    \StringTok{"GE"}\NormalTok{,}
    \StringTok{"INTC"}\NormalTok{,}
    \StringTok{"C"}\NormalTok{,}
    \StringTok{"GM"}
  \NormalTok{)}
\NormalTok{)}
\NormalTok{stocks <-}\StringTok{ }\KeywordTok{data.frame}\NormalTok{(}
  \NormalTok{stocks,}
  \KeywordTok{c}\NormalTok{(}
    \StringTok{"INTERNATIONAL BUSINESS MACHINES CORP"}\NormalTok{,}
    \StringTok{"EXXON MOBIL CORP"}\NormalTok{,}
    \StringTok{"CHEVRON CORP"}\NormalTok{,}
    \StringTok{"PROCTER & GAMBLE"}\NormalTok{,}
    \StringTok{"3M CO"}\NormalTok{,}
    \StringTok{"JOHNSON & JOHNSON"}\NormalTok{,}
    \StringTok{"MCDONALDS CORP"}\NormalTok{,}
    \StringTok{"WAL MART STORES INC"}\NormalTok{,}
    \StringTok{"UNITED TECHNOLOGIES CORP"}\NormalTok{,}
    \StringTok{"COCA COLA CO"}\NormalTok{,}
    \StringTok{"BOEING CO"}\NormalTok{,}
    \StringTok{"CATERPILLAR INC"}\NormalTok{,}
    \StringTok{"JPMORGAN CHASE & CO"}\NormalTok{,}
    \StringTok{"VERIZON COMMUNICATIONS INC"}\NormalTok{,}
    \StringTok{"AT&T INC."}\NormalTok{,}
    \StringTok{"DUPONT E I DE NEMOURS & CO"}\NormalTok{,}
    \StringTok{"Merck & Co. Inc."}\NormalTok{,}
    \StringTok{"WALT DISNEY CO/"}\NormalTok{,}
    \StringTok{"HOME DEPOT INC"}\NormalTok{,}
    \StringTok{"MICROSOFT CORP"}\NormalTok{,}
    \StringTok{"AMERICAN EXPRESS CO"}\NormalTok{,}
    \StringTok{"BANK OF AMERICA CORP /DE/"}\NormalTok{,}
    \StringTok{"PFIZER INC"}\NormalTok{,}
    \StringTok{"GENERAL ELECTRIC CO"}\NormalTok{,}
    \StringTok{"INTEL CORP"}\NormalTok{,}
    \StringTok{"CITIGROUP INC"}\NormalTok{,}
    \StringTok{"General Motors Co"}
  \NormalTok{)}
\NormalTok{)}
\KeywordTok{colnames}\NormalTok{(stocks) <-}\StringTok{ }\KeywordTok{c}\NormalTok{(}\StringTok{"ticker"}\NormalTok{, }\StringTok{"company"}\NormalTok{)}
\end{Highlighting}
\end{Shaded}

The code below imports data from yahoo finance website and download
historical prices for previously selected stocks, consolidating all
information in a single table.\textbf{Note}: At the time of writing this
paper, this code was functional. However, should yahoo decide to change
the website interface, the code may need tweeks to work correctly. The
dates have been hardcoded in the section below (\&a=8\&b=1\&c=2011 means
August 1st, 2011). They can be changed to use other periods. In this
research, the field adjusted closing (Adj Closing) price was the one
selected for the price change calculations.

\begin{Shaded}
\begin{Highlighting}[]
\NormalTok{firstRun =}\StringTok{ }\OtherTok{TRUE}
\NormalTok{for (t in stocks$ticker) \{}
  \NormalTok{URL <-}
\StringTok{    }\KeywordTok{paste}\NormalTok{(}
      \StringTok{"http://chart.finance.yahoo.com/table.csv?s="}\NormalTok{,}
      \NormalTok{t,}
      \StringTok{"&a=8&b=1&c=2011&d=7&e=31&f=2016&g=d&ignore=.csv"}\NormalTok{,}
      \CommentTok{# Hard-coded dates}
      \DataTypeTok{sep =} \StringTok{""}
    \NormalTok{)}
  \NormalTok{tmp <-}
\StringTok{    }\KeywordTok{fread}\NormalTok{(URL,}
          \DataTypeTok{drop =} \KeywordTok{c}\NormalTok{(}\StringTok{"Open"}\NormalTok{, }\StringTok{"High"}\NormalTok{, }\StringTok{"Low"}\NormalTok{, }\StringTok{"Close"}\NormalTok{, }\StringTok{"Volume"}\NormalTok{))}
  \NormalTok{tmp <-}\StringTok{ }\KeywordTok{data.frame}\NormalTok{(tmp, t, }\DecValTok{0}\NormalTok{)}
  \KeywordTok{colnames}\NormalTok{(tmp) <-}\StringTok{ }\KeywordTok{c}\NormalTok{(}\StringTok{"date"}\NormalTok{, }\StringTok{"price"}\NormalTok{, }\StringTok{"ticker"}\NormalTok{, }\StringTok{"flag"}\NormalTok{)}
  \NormalTok{if (firstRun ==}\StringTok{ }\OtherTok{TRUE}\NormalTok{) \{}
    \NormalTok{historicalPrices <-}\StringTok{ }\KeywordTok{data.frame}\NormalTok{(tmp)}
    \NormalTok{firstRun <-}\StringTok{ }\OtherTok{FALSE}
  \NormalTok{\} else}
    \NormalTok{historicalPrices <-}\StringTok{ }\KeywordTok{rbind}\NormalTok{(historicalPrices, tmp)}
\NormalTok{\}}
\KeywordTok{rm}\NormalTok{(tmp)}
\KeywordTok{rm}\NormalTok{(firstRun)}
\KeywordTok{rm}\NormalTok{(t)}
\KeywordTok{rm}\NormalTok{(URL)}
\end{Highlighting}
\end{Shaded}

There wasn't a good free formatted online source for daily US inflation
rates so it was manually estimated using excel and distributing the
monthly rate daily, including weekends and holidays. These daily rates
are used to adjust stock prices for inflation. The method used is to
change the daily price by daily rate (For example, if the stock price
was \$100 and inflation rate was 1\%, the adjusted price would be
\$101). In practice, the inflation rate didn't change the values
significantly for the periods considered.

\begin{Shaded}
\begin{Highlighting}[]
\NormalTok{fn <-}\StringTok{ "dailyInflationUS.xlsx"}
  
\NormalTok{if (!}\KeywordTok{file.exists}\NormalTok{(fn)) \{}
  \KeywordTok{download.file}\NormalTok{(URL, fn, }\DataTypeTok{method=}\StringTok{"auto"}\NormalTok{, }\DataTypeTok{cacheOK =} \OtherTok{FALSE}\NormalTok{)}
  \NormalTok{\}}
  
\NormalTok{inflationRates <-}
\StringTok{  }\KeywordTok{read.xlsx}\NormalTok{(}
  \NormalTok{fn,}
  \DataTypeTok{sheetIndex =} \DecValTok{1}\NormalTok{,}
  \DataTypeTok{colIndex =} \KeywordTok{c}\NormalTok{(}\DecValTok{1}\NormalTok{, }\DecValTok{4}\NormalTok{),}
  \DataTypeTok{stringsAsFactors =} \OtherTok{FALSE}
  \NormalTok{)}
\end{Highlighting}
\end{Shaded}

\textbf{Note}:The table used can be downloaded from
\url{https://drive.google.com/open?id=0B2BIWokA6cNdM0RmQ1k1eFlaQWs}

The code below consolidates the historical prices and US inflation rates
in the same file, excluding all the dates that that are outside the
interest range.

\begin{Shaded}
\begin{Highlighting}[]
\NormalTok{tmp <-}
\StringTok{  }\KeywordTok{merge}\NormalTok{(historicalPrices,}
        \NormalTok{inflationRates,}
        \DataTypeTok{by =} \StringTok{"date"}\NormalTok{,}
        \DataTypeTok{all =} \OtherTok{TRUE}\NormalTok{)}
\NormalTok{tmp$date <-}\StringTok{ }\KeywordTok{as.Date}\NormalTok{(tmp$date, }\StringTok{"%Y-%m-%d"}\NormalTok{)}
\NormalTok{tmp <-}
\StringTok{  }\KeywordTok{subset}\NormalTok{(tmp,}
         \NormalTok{date >=}\StringTok{ "2011/09/01"} \NormalTok{&}
\StringTok{           }\NormalTok{date <=}\StringTok{ "2016/07/31"} \NormalTok{&}
\StringTok{           }\NormalTok{!}\KeywordTok{is.na}\NormalTok{(ticker)}
         \NormalTok{)}
\NormalTok{historicalPricesInflation <-}\StringTok{ }\NormalTok{tmp}
\KeywordTok{rm}\NormalTok{(tmp)}
\KeywordTok{rm}\NormalTok{(inflationRates)}
\KeywordTok{rm}\NormalTok{(historicalPrices)}
\end{Highlighting}
\end{Shaded}

Next, the price changes are calculated by the formula (d + 1 - d) / (d),
where d is the price on a specific day. In sequence, the price changes
are then normalized to the normal distribution.

\begin{Shaded}
\begin{Highlighting}[]
\NormalTok{tmp <-}\StringTok{ }\KeywordTok{setorder}\NormalTok{(historicalPricesInflation,}
                \StringTok{"ticker"}\NormalTok{,}
                \StringTok{"date"}
                \NormalTok{)}
\NormalTok{tmp <-}\StringTok{ }\KeywordTok{data.frame}\NormalTok{(tmp, }\DecValTok{0}\NormalTok{)}
\KeywordTok{colnames}\NormalTok{(tmp)[}\DecValTok{6}\NormalTok{] <-}\StringTok{ "change"}
  
\NormalTok{for (i in }\DecValTok{1}\NormalTok{:}\KeywordTok{nrow}\NormalTok{(tmp)) \{}
  \NormalTok{if (i ==}\StringTok{ }\DecValTok{1} \NormalTok{||}\StringTok{ }\NormalTok{tmp[i,}\StringTok{"ticker"}\NormalTok{] !=}\StringTok{ }\NormalTok{tmp[i -}\StringTok{ }\DecValTok{1}\NormalTok{,}\StringTok{"ticker"}\NormalTok{] ) \{}
    \NormalTok{tmp[i, }\StringTok{"change"}\NormalTok{] <-}\StringTok{ }\OtherTok{NA}
  \NormalTok{\} else \{}
    \NormalTok{tmp[i, }\StringTok{"change"}\NormalTok{] <-}
\StringTok{      }\NormalTok{(tmp[i, }\StringTok{"price"}\NormalTok{] -}
\StringTok{         }\NormalTok{tmp[i -}\StringTok{ }\DecValTok{1}\NormalTok{, }\StringTok{"price"}\NormalTok{] *}
\StringTok{         }\NormalTok{(}\DecValTok{1} \NormalTok{+}\StringTok{ }\NormalTok{tmp[i, }\StringTok{"rate"}\NormalTok{])) /}
\StringTok{      }\NormalTok{tmp[i}\DecValTok{-1}\NormalTok{,}\StringTok{"price"}\NormalTok{]*(}\DecValTok{1} \NormalTok{+}\StringTok{ }\NormalTok{tmp[i, }\StringTok{"rate"}\NormalTok{])}
  \NormalTok{\}}
\NormalTok{\}}
\NormalTok{adjustedPriceChanges <-}\StringTok{ }\KeywordTok{subset}\NormalTok{(tmp,!}\KeywordTok{is.na}\NormalTok{(change))}
\KeywordTok{rm}\NormalTok{(i)}
\KeywordTok{rm}\NormalTok{(tmp)}

\NormalTok{for (t in stocks$ticker) \{}
  \NormalTok{tmp <-}\StringTok{ }\KeywordTok{subset}\NormalTok{(adjustedPriceChanges, ticker ==}\StringTok{ }\NormalTok{t)}
  \NormalTok{tmp$change <-}\StringTok{ }\KeywordTok{scale}\NormalTok{(tmp$change)}
  \NormalTok{if (!}\KeywordTok{exists}\NormalTok{(}\StringTok{"scaledPriceChanges"}\NormalTok{))}
    \NormalTok{scaledPriceChanges <-}\StringTok{ }\NormalTok{tmp}
  \NormalTok{else}
    \NormalTok{scaledPriceChanges <-}\StringTok{ }\KeywordTok{rbind}\NormalTok{(scaledPriceChanges, tmp)}
\NormalTok{\}}
\KeywordTok{rm}\NormalTok{(t)}
\end{Highlighting}
\end{Shaded}

The following step captures filing dates from SEC.org website in the
period considered (the period can be changed in the two
\emph{\enquote{for}} lines, where \textbf{i} is year and \textbf{j} is
quarter). As now the code doesn't check for the filing existence so in
the case it doesn't, the script will return an error. Next, consolidate
all lines together, filtering date by filing type, 10-Q and 10-K, and by
companies, the ones declared at the beggining. There is a known issue:
since the names companies use to file their financial statements can
have small changes, the operator \%like\% is used so that the script can
look inside the fields for contained text instead of equal text. The
problem with this approach is that sometimes it brings up more than one
company, which needs separated treatment.

\begin{Shaded}
\begin{Highlighting}[]
\NormalTok{firstRun <-}\StringTok{ }\OtherTok{TRUE}
\NormalTok{for (i in }\DecValTok{2011}\NormalTok{:}\DecValTok{2016}\NormalTok{) \{}
  \NormalTok{for (j in }\DecValTok{1}\NormalTok{:}\DecValTok{4}\NormalTok{) \{}
    \NormalTok{if (!(i ==}\StringTok{ }\DecValTok{2016} \NormalTok{&}\StringTok{ }\NormalTok{j ==}\StringTok{ }\DecValTok{4}\NormalTok{)) \{}
      \NormalTok{URL <-}
\StringTok{        }\KeywordTok{paste}\NormalTok{(}\StringTok{"ftp://ftp.sec.gov/edgar/full-index/"}\NormalTok{,}
              \NormalTok{i,}
              \StringTok{"/QTR"}\NormalTok{,}
              \NormalTok{j,}
              \StringTok{"/master.zip"}\NormalTok{,}
              \DataTypeTok{sep =} \StringTok{""}\NormalTok{)}
      \NormalTok{FNC <-}\StringTok{ }\KeywordTok{paste}\NormalTok{(}\StringTok{"master_QTR"}\NormalTok{, j, }\StringTok{"_"}\NormalTok{, i, }\StringTok{".zip"}\NormalTok{, }\DataTypeTok{sep =} \StringTok{""}\NormalTok{)}
      \NormalTok{FNU <-}\StringTok{ }\KeywordTok{paste}\NormalTok{(}\StringTok{"master_QTR"}\NormalTok{, j, }\StringTok{"_"}\NormalTok{, i, }\StringTok{".idx"}\NormalTok{, }\DataTypeTok{sep =} \StringTok{""}\NormalTok{)}
      \NormalTok{if (!}\KeywordTok{file.exists}\NormalTok{(FNC))}
        \KeywordTok{download.file}\NormalTok{(URL, FNC, }\DataTypeTok{method =} \StringTok{"libcurl"}\NormalTok{)}
      \KeywordTok{unzip}\NormalTok{(FNC, }\StringTok{"master.idx"}\NormalTok{)}
      \KeywordTok{file.rename}\NormalTok{(}\StringTok{"master.idx"}\NormalTok{, FNU)}
      \NormalTok{if (firstRun ==}\StringTok{ }\OtherTok{TRUE}\NormalTok{) \{}
        \NormalTok{filingDates <-}\StringTok{ }\KeywordTok{fread}\NormalTok{(FNU)}
        \NormalTok{filingDates <-}
\StringTok{          }\KeywordTok{subset}\NormalTok{(filingDates, V3 ==}\StringTok{ "10-K"} \NormalTok{|}\StringTok{ }\NormalTok{V3 ==}\StringTok{ "10-Q"}\NormalTok{)}
        \NormalTok{firstRun <-}\StringTok{ }\OtherTok{FALSE}
      \NormalTok{\} else \{}
        \NormalTok{tmp <-}\StringTok{ }\KeywordTok{fread}\NormalTok{(FNU)}
        \NormalTok{tmp <-}\StringTok{ }\KeywordTok{subset}\NormalTok{(tmp, V3 ==}\StringTok{ "10-K"} \NormalTok{|}\StringTok{ }\NormalTok{V3 ==}\StringTok{ "10-Q"}\NormalTok{)}
        \NormalTok{filingDates <-}\StringTok{ }\KeywordTok{rbind}\NormalTok{(filingDates, tmp)}
      \NormalTok{\}}
    \NormalTok{\}}
  \NormalTok{\}}
\NormalTok{\}}
\KeywordTok{colnames}\NormalTok{(filingDates) <-}\StringTok{ }\KeywordTok{c}\NormalTok{(}\StringTok{"CIK"}\NormalTok{,}\StringTok{"company"}\NormalTok{,}\StringTok{"filing"}\NormalTok{,}\StringTok{"date"}\NormalTok{,}\StringTok{"location"}\NormalTok{)}
\KeywordTok{rm}\NormalTok{(i)}
\KeywordTok{rm}\NormalTok{(j)}
\KeywordTok{rm}\NormalTok{(URL)}
\KeywordTok{rm}\NormalTok{(firstRun)}
\KeywordTok{rm}\NormalTok{(tmp)}

\NormalTok{for (n in }\DecValTok{1}\NormalTok{:}\KeywordTok{nrow}\NormalTok{(stocks)) \{}
  \NormalTok{tmp <-}
\StringTok{    }\KeywordTok{data.frame}\NormalTok{(}\KeywordTok{subset}\NormalTok{(filingDates, company %like%}\StringTok{ }\NormalTok{stocks$company[n]),}
    \NormalTok{stocks$ticker[n])}
  \NormalTok{if (!}\KeywordTok{exists}\NormalTok{(}\StringTok{"relevantFilings"}\NormalTok{))}
    \NormalTok{relevantFilings <-}\StringTok{ }\NormalTok{tmp}
  \NormalTok{else}
    \NormalTok{relevantFilings <-}\StringTok{ }\KeywordTok{rbind}\NormalTok{(relevantFilings, tmp)}
\NormalTok{\}}
\end{Highlighting}
\end{Shaded}

This is the place to treat the exceptions above. In this example, four
addional companies were brought in the sample which are not related to
the analysis and must be eliminated. \textbf{Note}: If other stocks are
added, this section may need to include other incorrectly included
companies.

\begin{Shaded}
\begin{Highlighting}[]
\NormalTok{relevantFilings <-}
\StringTok{  }\KeywordTok{subset}\NormalTok{(}
  \NormalTok{relevantFilings,}
  \NormalTok{company !=}\StringTok{ "STRUCTURED PRODUCTS CORP CORTS TRUST FOR BOEING CO NOTES"} \NormalTok{&}
\StringTok{  }\NormalTok{company !=}\StringTok{ "STRATS SM TRUST FOR JPMORGAN CHASE & CO SEC SERIES 2004-9"} \NormalTok{&}
\StringTok{  }\NormalTok{company !=}\StringTok{ "PORTLAND GENERAL ELECTRIC CO /OR/"} \NormalTok{&}
\StringTok{  }\NormalTok{company !=}\StringTok{ "CINTEL CORP"}
  \NormalTok{)}
\end{Highlighting}
\end{Shaded}

The section below joins the data from the filings and from historical
pricing.

\begin{Shaded}
\begin{Highlighting}[]
\NormalTok{relevantFilings <-}
\StringTok{  }\KeywordTok{data.frame}\NormalTok{(}\KeywordTok{as.Date}\NormalTok{(relevantFilings$date, }\StringTok{"%Y-%m-%d"}\NormalTok{),}
  \NormalTok{relevantFilings$stocks.ticker.n.,}
  \DecValTok{1}\NormalTok{)}
\KeywordTok{colnames}\NormalTok{(relevantFilings) <-}\StringTok{ }\KeywordTok{c}\NormalTok{(}\StringTok{"date"}\NormalTok{, }\StringTok{"ticker"}\NormalTok{, }\StringTok{"flag"}\NormalTok{)}
\NormalTok{scaledPriceChanges <-}
\StringTok{  }\KeywordTok{data.frame}\NormalTok{(scaledPriceChanges$date,}
  \NormalTok{scaledPriceChanges$ticker,}
  \NormalTok{scaledPriceChanges$change)}
\KeywordTok{colnames}\NormalTok{(scaledPriceChanges) <-}\StringTok{ }\KeywordTok{c}\NormalTok{(}\StringTok{"date"}\NormalTok{, }\StringTok{"ticker"}\NormalTok{, }\StringTok{"change"}\NormalTok{)}

\NormalTok{finalData <-}
\StringTok{  }\KeywordTok{merge}\NormalTok{(}
  \NormalTok{scaledPriceChanges,}
  \NormalTok{relevantFilings,}
  \DataTypeTok{by =} \KeywordTok{c}\NormalTok{(}\StringTok{"date"}\NormalTok{, }\StringTok{"ticker"}\NormalTok{),}
  \DataTypeTok{all.x =} \OtherTok{TRUE}
  \NormalTok{)}
\NormalTok{for (i in }\DecValTok{1}\NormalTok{:}\KeywordTok{nrow}\NormalTok{(finalData)) \{}
  \NormalTok{if (}\KeywordTok{is.na}\NormalTok{(finalData$flag[i]))}
    \NormalTok{finalData$flag[i] <-}\StringTok{ }\DecValTok{0}
\NormalTok{\}}
\end{Highlighting}
\end{Shaded}

This section marks the days that the dates affected by the
\enquote{Release} of the filings. For this project, it was considered
day - 2 to day + 2. These days are marked and will constitute the subset
called \enquote{Release} in the final analysis. The remaining days will
be treated as \enquote{Regular}days. After this last step, the data will
be ready for analysis.

\section{Results}\label{results}

Below the variance of both subsets is shown.

\begin{verbatim}
## [1] "Regular set variance: 0.99"
\end{verbatim}

\begin{verbatim}
## [1] "Release set variance: 1.08"
\end{verbatim}

On the same data, a hypothesis test for the two populations to have the
same variance is ran:

\begin{verbatim}
## 
##  F test to compare two variances
## 
## data:  finalData.reg and finalData.fin
## F = 0.9, num df = 30000, denom df = 3000, p-value = 0.004
## alternative hypothesis: true ratio of variances is not equal to 1
## 95 percent confidence interval:
##  0.87 0.97
## sample estimates:
## ratio of variances 
##               0.92
\end{verbatim}

The graphs of the distribution densities are shown in Figure 1 through 4
for visual comparison of the two samples.

\section{Discussion}\label{discussion}

The objective of this study was to show the existance or not of any
relation between price change variance. To this end, the variance of
both subsets were shown and a basic variance test was used to compare
the two populations. Results above shows that not only the variance of
the two populations were different, with the \enquote{Release} set being
higher than the \enquote{Regular} but the p-value from the hypothesis
test allows us to reject the main hypothesis that the two populations
have the same variance. Figures 1 to 4 show that while the distribution
around the center is similar for both populations, the distribution gets
very different in the tails especially below -5 and above 5. These
results suggest a potential strategy for option straddle for two main
reasons. First, there is indeed a greater variation close to the release
date, important for neutral straddles. Secondly, difference is
notoriously on the graphs tails, as the cost to form a straddle with
puts and calls with strike price far from the current are significantly
cheaper, increasing the potential for returns. In summary, the results
shown here provide the basis for the next steps: to check the price
variance in time, to collect options data on analyzed stocks and to run
simulations for the different possible straddle strategies. If the
results are optmistic still optimistic, re-running the analysis on a
different set of stocks for backtesting would confirm the feasibly of
the strategy.

\newpage

\section{References}\label{references}

\begin{figure}[htbp]
\centering
\includegraphics{G7_Final_Paper_files/figure-latex/unnamed-chunk-2-1.pdf}
\caption{\label{fig:unnamed-chunk-2}Superposed distribution density of the
two datasets.}
\end{figure}

\begin{figure}[htbp]
\centering
\includegraphics{G7_Final_Paper_files/figure-latex/unnamed-chunk-3-1.pdf}
\caption{\label{fig:unnamed-chunk-3}Distribution density zoomed around the
center.}
\end{figure}

\begin{figure}[htbp]
\centering
\includegraphics{G7_Final_Paper_files/figure-latex/unnamed-chunk-4-1.pdf}
\caption{\label{fig:unnamed-chunk-4}Distribution density zoomed on the left
tail.}
\end{figure}

\begin{figure}[htbp]
\centering
\includegraphics{G7_Final_Paper_files/figure-latex/unnamed-chunk-5-1.pdf}
\caption{\label{fig:unnamed-chunk-5}Distribution density zoomed on the right
tail.}
\end{figure}

\section{Session Info}\label{session-info}

\begin{Shaded}
\begin{Highlighting}[]
\KeywordTok{sessionInfo}\NormalTok{()}
\end{Highlighting}
\end{Shaded}

\begin{verbatim}
## R version 3.3.2 (2016-10-31)
## Platform: x86_64-w64-mingw32/x64 (64-bit)
## Running under: Windows 7 x64 (build 7601) Service Pack 1
## 
## locale:
## [1] LC_COLLATE=English_United States.1252 
## [2] LC_CTYPE=English_United States.1252   
## [3] LC_MONETARY=English_United States.1252
## [4] LC_NUMERIC=C                          
## [5] LC_TIME=English_United States.1252    
## 
## attached base packages:
## [1] stats     graphics  grDevices utils     datasets  methods   base     
## 
## other attached packages:
## [1] ggplot2_2.2.1     xlsx_0.5.7        xlsxjars_0.6.1    rJava_0.9-8      
## [5] data.table_1.10.0 papaja_0.1.0.9477
## 
## loaded via a namespace (and not attached):
##  [1] Rcpp_0.12.9      knitr_1.15.1     magrittr_1.5     munsell_0.4.3   
##  [5] colorspace_1.3-2 highr_0.6        stringr_1.1.0    plyr_1.8.4      
##  [9] tools_3.3.2      grid_3.3.2       gtable_0.2.0     htmltools_0.3.5 
## [13] assertthat_0.1   yaml_2.1.14      lazyeval_0.2.0   rprojroot_1.2   
## [17] digest_0.6.12    tibble_1.2       bookdown_0.3     evaluate_0.10   
## [21] rmarkdown_1.3    labeling_0.3     stringi_1.1.2    scales_0.4.1    
## [25] backports_1.0.5
\end{verbatim}

\setlength{\parindent}{-0.5in} \setlength{\leftskip}{0.5in}

\hypertarget{refs}{}
\hypertarget{ref-R-papaja}{}
Aust, F., \& Barth, M. (2016). \emph{Papaja: Create apa manuscripts with
rmarkdown}. Retrieved from \url{https://github.com/crsh/papaja}

\hypertarget{ref-berkdemarzo}{}
Berk, P. M., Jonathan BDeMarzo. (n.d.). \emph{Corporate finance}.

\hypertarget{ref-bodie}{}
Bodie, Z., Kane, A., \& Marcus, A. J. (n.d.). \emph{Investments}.

\hypertarget{ref-R-data.table}{}
Dowle, M., Srinivasan, A., Short, T., R Saporta, S. L. with
contributions from, \& Antonyan, E. (2015). \emph{Data.table: Extension
of data.frame}. Retrieved from
\url{https://CRAN.R-project.org/package=data.table}

\hypertarget{ref-R-xlsx}{}
Dragulescu, A. A. (2014a). \emph{Xlsx: Read, write, format excel 2007
and excel 97/2000/xp/2003 files}. Retrieved from
\url{https://CRAN.R-project.org/package=xlsx}

\hypertarget{ref-R-xlsxjars}{}
Dragulescu, A. A. (2014b). \emph{Xlsxjars: Package required poi jars for
the xlsx package}. Retrieved from
\url{https://CRAN.R-project.org/package=xlsxjars}

\hypertarget{ref-straddle}{}
Investopedia. (2003). \emph{Straddle definition}. Retrieved from
\url{http://www.investopedia.com/terms/s/straddle.asp}

\hypertarget{ref-R-base}{}
R Core Team. (2016). \emph{R: A language and environment for statistical
computing}. Vienna, Austria: R Foundation for Statistical Computing.
Retrieved from \url{https://www.R-project.org/}

\hypertarget{ref-sec_1}{}
Sec.gov \textbar{} company search page\_2016. (2016). \emph{Sec.gov}.
Retrieved from
\url{https://www.sec.gov/edgar/searchedgar/companysearch.html}

\hypertarget{ref-sec_2}{}
Sec.org \textbar{} information for ftp users - 2016. (2016).
\emph{Sec.gov}. Retrieved from
\url{https://www.sec.gov/edgar/searchedgar/ftpusers.htm}

\hypertarget{ref-R-rJava}{}
Urbanek, S. (2016). \emph{RJava: Low-level r to java interface}.
Retrieved from \url{https://CRAN.R-project.org/package=rJava}

\hypertarget{ref-R-ggplot2}{}
Wickham, H. (2009). \emph{Ggplot2: Elegant graphics for data analysis}.
Springer-Verlag New York. Retrieved from \url{http://ggplot2.org}

\hypertarget{ref-yahoo_1}{}
Yahoo finance - business finance, stock market, quotes, news\_2016.
(2016). \emph{Finance.yahoo.com}. Retrieved from
\url{http://finance.yahoo.com/}






\end{document}
