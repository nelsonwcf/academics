\documentclass[english,man]{apa6}

\usepackage{amssymb,amsmath}
\usepackage{ifxetex,ifluatex}
\usepackage{fixltx2e} % provides \textsubscript
\ifnum 0\ifxetex 1\fi\ifluatex 1\fi=0 % if pdftex
  \usepackage[T1]{fontenc}
  \usepackage[utf8]{inputenc}
\else % if luatex or xelatex
  \ifxetex
    \usepackage{mathspec}
    \usepackage{xltxtra,xunicode}
  \else
    \usepackage{fontspec}
  \fi
  \defaultfontfeatures{Mapping=tex-text,Scale=MatchLowercase}
  \newcommand{\euro}{€}
\fi
% use upquote if available, for straight quotes in verbatim environments
\IfFileExists{upquote.sty}{\usepackage{upquote}}{}
% use microtype if available
\IfFileExists{microtype.sty}{\usepackage{microtype}}{}
\usepackage{color}
\usepackage{fancyvrb}
\newcommand{\VerbBar}{|}
\newcommand{\VERB}{\Verb[commandchars=\\\{\}]}
\DefineVerbatimEnvironment{Highlighting}{Verbatim}{commandchars=\\\{\}}
% Add ',fontsize=\small' for more characters per line
\usepackage{framed}
\definecolor{shadecolor}{RGB}{248,248,248}
\newenvironment{Shaded}{\begin{snugshade}}{\end{snugshade}}
\newcommand{\KeywordTok}[1]{\textcolor[rgb]{0.13,0.29,0.53}{\textbf{{#1}}}}
\newcommand{\DataTypeTok}[1]{\textcolor[rgb]{0.13,0.29,0.53}{{#1}}}
\newcommand{\DecValTok}[1]{\textcolor[rgb]{0.00,0.00,0.81}{{#1}}}
\newcommand{\BaseNTok}[1]{\textcolor[rgb]{0.00,0.00,0.81}{{#1}}}
\newcommand{\FloatTok}[1]{\textcolor[rgb]{0.00,0.00,0.81}{{#1}}}
\newcommand{\ConstantTok}[1]{\textcolor[rgb]{0.00,0.00,0.00}{{#1}}}
\newcommand{\CharTok}[1]{\textcolor[rgb]{0.31,0.60,0.02}{{#1}}}
\newcommand{\SpecialCharTok}[1]{\textcolor[rgb]{0.00,0.00,0.00}{{#1}}}
\newcommand{\StringTok}[1]{\textcolor[rgb]{0.31,0.60,0.02}{{#1}}}
\newcommand{\VerbatimStringTok}[1]{\textcolor[rgb]{0.31,0.60,0.02}{{#1}}}
\newcommand{\SpecialStringTok}[1]{\textcolor[rgb]{0.31,0.60,0.02}{{#1}}}
\newcommand{\ImportTok}[1]{{#1}}
\newcommand{\CommentTok}[1]{\textcolor[rgb]{0.56,0.35,0.01}{\textit{{#1}}}}
\newcommand{\DocumentationTok}[1]{\textcolor[rgb]{0.56,0.35,0.01}{\textbf{\textit{{#1}}}}}
\newcommand{\AnnotationTok}[1]{\textcolor[rgb]{0.56,0.35,0.01}{\textbf{\textit{{#1}}}}}
\newcommand{\CommentVarTok}[1]{\textcolor[rgb]{0.56,0.35,0.01}{\textbf{\textit{{#1}}}}}
\newcommand{\OtherTok}[1]{\textcolor[rgb]{0.56,0.35,0.01}{{#1}}}
\newcommand{\FunctionTok}[1]{\textcolor[rgb]{0.00,0.00,0.00}{{#1}}}
\newcommand{\VariableTok}[1]{\textcolor[rgb]{0.00,0.00,0.00}{{#1}}}
\newcommand{\ControlFlowTok}[1]{\textcolor[rgb]{0.13,0.29,0.53}{\textbf{{#1}}}}
\newcommand{\OperatorTok}[1]{\textcolor[rgb]{0.81,0.36,0.00}{\textbf{{#1}}}}
\newcommand{\BuiltInTok}[1]{{#1}}
\newcommand{\ExtensionTok}[1]{{#1}}
\newcommand{\PreprocessorTok}[1]{\textcolor[rgb]{0.56,0.35,0.01}{\textit{{#1}}}}
\newcommand{\AttributeTok}[1]{\textcolor[rgb]{0.77,0.63,0.00}{{#1}}}
\newcommand{\RegionMarkerTok}[1]{{#1}}
\newcommand{\InformationTok}[1]{\textcolor[rgb]{0.56,0.35,0.01}{\textbf{\textit{{#1}}}}}
\newcommand{\WarningTok}[1]{\textcolor[rgb]{0.56,0.35,0.01}{\textbf{\textit{{#1}}}}}
\newcommand{\AlertTok}[1]{\textcolor[rgb]{0.94,0.16,0.16}{{#1}}}
\newcommand{\ErrorTok}[1]{\textcolor[rgb]{0.64,0.00,0.00}{\textbf{{#1}}}}
\newcommand{\NormalTok}[1]{{#1}}

% Table formatting
\usepackage{longtable, booktabs}
\usepackage{lscape}
% \usepackage[counterclockwise]{rotating}   % Landscape page setup for large tables
\usepackage{multirow}		% Table styling
\usepackage{tabularx}		% Control Column width
\usepackage[flushleft]{threeparttable}	% Allows for three part tables with a specified notes section
\usepackage{threeparttablex}            % Lets threeparttable work with longtable

% Create new environments so endfloat can handle them
\newenvironment{ltable}
  {\begin{landscape}\begin{center}\begin{threeparttable}}
  {\end{threeparttable}\end{center}\end{landscape}}

\newenvironment{lltable}
  {\begin{landscape}\begin{center}\begin{ThreePartTable}}
  {\end{ThreePartTable}\end{center}\end{landscape}}

\usepackage{ifthen} % Only add declarations when endfloat package is loaded
\ifthenelse{\equal{\string man}{\string man}}{%
 \DeclareDelayedFloatFlavor{ThreePartTable}{table} % Make endfloat play with longtable
 \DeclareDelayedFloatFlavor{ltable}{table} % Make endfloat play with lscape
 \DeclareDelayedFloatFlavor{lltable}{table} % Make endfloat play with lscape & longtable
}{}%


% The following enables adjusting longtable caption width to table width
% Solution found at http://golatex.de/longtable-mit-caption-so-breit-wie-die-tabelle-t15767.html
\makeatletter
\newcommand\LastLTentrywidth{1em}
\newlength\longtablewidth
\setlength{\longtablewidth}{1in}
\newcommand\getlongtablewidth{%
 \begingroup
  \ifcsname LT@\roman{LT@tables}\endcsname
  \global\longtablewidth=0pt
  \renewcommand\LT@entry[2]{\global\advance\longtablewidth by ##2\relax\gdef\LastLTentrywidth{##2}}%
  \@nameuse{LT@\roman{LT@tables}}%
  \fi
\endgroup}


  \usepackage{graphicx}
  \makeatletter
  \def\maxwidth{\ifdim\Gin@nat@width>\linewidth\linewidth\else\Gin@nat@width\fi}
  \def\maxheight{\ifdim\Gin@nat@height>\textheight\textheight\else\Gin@nat@height\fi}
  \makeatother
  % Scale images if necessary, so that they will not overflow the page
  % margins by default, and it is still possible to overwrite the defaults
  % using explicit options in \includegraphics[width, height, ...]{}
  \setkeys{Gin}{width=\maxwidth,height=\maxheight,keepaspectratio}
\ifxetex
  \usepackage[setpagesize=false, % page size defined by xetex
              unicode=false, % unicode breaks when used with xetex
              xetex]{hyperref}
\else
  \usepackage[unicode=true]{hyperref}
\fi
\hypersetup{breaklinks=true,
            pdfauthor={},
            pdftitle={Options Straddle Simulations - Effects of Financial Reports on Straddle Strategies},
            colorlinks=true,
            citecolor=blue,
            urlcolor=blue,
            linkcolor=black,
            pdfborder={0 0 0}}
\urlstyle{same}  % don't use monospace font for urls

\setlength{\parindent}{0pt}
%\setlength{\parskip}{0pt plus 0pt minus 0pt}

\setlength{\emergencystretch}{3em}  % prevent overfull lines

\ifxetex
  \usepackage{polyglossia}
  \setmainlanguage{}
\else
  \usepackage[english]{babel}
\fi

% Manuscript styling
\captionsetup{font=singlespacing,justification=justified}
\usepackage{csquotes}
\usepackage{upgreek}



\usepackage{tikz} % Variable definition to generate author note

% fix for \tightlist problem in pandoc 1.14
\providecommand{\tightlist}{%
  \setlength{\itemsep}{0pt}\setlength{\parskip}{0pt}}

% Essential manuscript parts
  \title{Options Straddle Simulations - Effects of Financial Reports on Straddle
Strategies}

  \shorttitle{STRADDLE SIMULATIONS USING FINANCIAL REPORTS DATES}


  \author{Nelson Corrocher~\& Sergii Savchuk}

  \def\affdep{{"", ""}}%
  \def\affcity{{"", ""}}%

  \affiliation{
    \vspace{0.5cm}
          \textsuperscript{} Harrisburg University  }

  \authornote{
    \newcounter{author}
    Nelson Corrocher, M.B.A.

                      Correspondence concerning this article should be addressed to Nelson Corrocher. E-mail: \href{mailto:ncorrocher@my.harrisburgu.edu}{\nolinkurl{ncorrocher@my.harrisburgu.edu}}
                          }


  \abstract{This paper continues the analysis of the differences between stock price
variation that occurs regularly and the variation that happens close to
10-Q and 10-K filing dates. After concluding that the variance of
blue-chips stock prices is indeed higher in the D-2 to D+2 on the
financial report release dates, the next step will be to collect stock
options prices from release date to M+3 months after the release date
and use this information to simulate what profits could be achieved if a
straddle was used close to the release date of the financial report.}
  \keywords{straddle options financial reports price variance simulation
black-scholes \\

    \indent Word count: 137
  }





\begin{document}

\maketitle

\setcounter{secnumdepth}{0}



\textbf{Note}: the SETWD function in the block above sets the working
directory and should be set to the folder where the inputs and outputs
would be located.

During a trading simulation in a graduate investment class at Boston
University in 2014, the winning strategy used an option straddle close
to the release date of the 10-K of a certain public company. This
outcome started a discussion of using a systematic creation of straddles
close to filing dates of companies. In the context of financial
investment, a straddle is a neutral option strategy involving the
simultaneous buying of a put and a call of the same underlying stock,
expecting a great price variation before the option expiration date.
Creating an option straddle close to the filing date of the financial
statement from publicly traded companies has been used by fund managers
to try to improve their funds returns. However, since they use
additional and often undisclosed criteria to decide whose company's
stocks they straddle, there is neither evidence that the variance is
indeed greater close to financial report filing dates, nor that the
strategy, by itself, is profitable. This research is a continuation of
the previous paper in which Blue Chip stock variance were calculated in
what we called \enquote{Regular}" days and \enquote{Release} days (days
in which Financial Statements are released in the market, defined, for
this research, as D - 2 to D + 2 with D being the day of the release).
In the previous research, it was found that in Blue Chip stocks, the
variance was different, and the difference could be seen at the
superposed distributions especially on the tails. This difference
suggests a potential strategy using straddles and such analysis is the
goal of this paper. From a higher perspective, the steps for this
project involve collecting historical options data and creating a
straddle strategy in the respective stock close to the days the
financial statement is going to be released and estimate the returns on
such a stragey. However, due to restrictions in the availability of
historical options price data, the feasibility of the strategy itself
won't be validated in this paper. Only the methodology will be
presented.

\section{Methods}\label{methods}

\subsection{Participants}\label{participants}

No direct participants were involved in this research as the main inputs
for the analysis were financial market data that are publicly available.

\subsection{Measures and Procedures}\label{measures-and-procedures}

This entire paper has been constructed almost entirely by using R
scripting (exception for Inflation Daily Rates, explained below). For
replicability purposes, the methodology is going to be explained through
the commented script, enabling the reader to change it for his own
needs. There were some unexpected difficulties during the study due the
fact that historical option prices is not freely available. This
restricted the ability to run collect updated data automatically and
reduced the scope of the script. Now, while the methodology will be
provided in a replicable way, it will be applied only to a sample of the
data, which won't be enough to validate the options strategy. The code
below is used to prepare the environment for the script.

\begin{Shaded}
\begin{Highlighting}[]
\NormalTok{list.of.packages <-}\StringTok{ }\KeywordTok{c}\NormalTok{(}\StringTok{"ggplot2"}\NormalTok{, }\StringTok{"xlsx"}\NormalTok{, }\StringTok{"data.table"}\NormalTok{)}
\NormalTok{for(i in list.of.packages) \{}
  \NormalTok{if(!}\KeywordTok{is.installed}\NormalTok{(i)) \{}
  \KeywordTok{install.packages}\NormalTok{(i)}
  \NormalTok{\}}
\NormalTok{\}}
\KeywordTok{library}\NormalTok{(data.table)}
\KeywordTok{library}\NormalTok{(xlsx)}
\KeywordTok{library}\NormalTok{(ggplot2)}
\end{Highlighting}
\end{Shaded}

As mentioned above, it won't be possible to automate the collection of
the data like was done in the previous paper. Instead, we will focus of
the treatment of the data once the data has been collected and put in
the working directory. The data used in this document has been pruchased
from the Chicago Board Options Exchange (CBOE) and uses its format. In
this example, we are going to use the ticker \enquote{IBM} since it is
the one that has historical options data. The code below comes mostly
from the previous paper and it is deeply explained there.

\begin{Shaded}
\begin{Highlighting}[]
\NormalTok{stocks <-}\StringTok{ }\KeywordTok{data.frame}\NormalTok{(}\StringTok{"IBM"}\NormalTok{,}\StringTok{"INTERNATIONAL BUSINESS MACHINES CORP"}\NormalTok{)}
\KeywordTok{colnames}\NormalTok{(stocks) <-}\StringTok{ }\KeywordTok{c}\NormalTok{(}\StringTok{"ticker"}\NormalTok{, }\StringTok{"company"}\NormalTok{)}

\NormalTok{firstRun =}\StringTok{ }\OtherTok{TRUE}
\NormalTok{for (t in stocks$ticker) \{}
  \NormalTok{URL <-}\StringTok{ }\KeywordTok{paste}\NormalTok{(}\StringTok{"http://chart.finance.yahoo.com/table.csv?s="}\NormalTok{,t,}\StringTok{"&a=8&b=1&c=2011&d=7&e=31&f=2016&g=d&ignore=.csv"}\NormalTok{,}\DataTypeTok{sep =} \StringTok{""}\NormalTok{)}
  \NormalTok{tmp <-}\StringTok{ }\KeywordTok{fread}\NormalTok{(URL, }\DataTypeTok{drop =} \KeywordTok{c}\NormalTok{(}\StringTok{"Open"}\NormalTok{, }\StringTok{"High"}\NormalTok{, }\StringTok{"Low"}\NormalTok{, }\StringTok{"Close"}\NormalTok{, }\StringTok{"Volume"}\NormalTok{))}
  \NormalTok{tmp <-}\StringTok{ }\KeywordTok{data.frame}\NormalTok{(tmp, t, }\DecValTok{0}\NormalTok{)}
  \KeywordTok{colnames}\NormalTok{(tmp) <-}\StringTok{ }\KeywordTok{c}\NormalTok{(}\StringTok{"date"}\NormalTok{, }\StringTok{"price"}\NormalTok{, }\StringTok{"ticker"}\NormalTok{, }\StringTok{"flag"}\NormalTok{)}
  \NormalTok{if (firstRun ==}\StringTok{ }\OtherTok{TRUE}\NormalTok{) \{}
    \NormalTok{historicalPrices <-}\StringTok{ }\KeywordTok{data.frame}\NormalTok{(tmp)}
    \NormalTok{firstRun <-}\StringTok{ }\OtherTok{FALSE}
  \NormalTok{\} else}
    \NormalTok{historicalPrices <-}\StringTok{ }\KeywordTok{rbind}\NormalTok{(historicalPrices, tmp)}
\NormalTok{\}}

\KeywordTok{rm}\NormalTok{(tmp)}
\KeywordTok{rm}\NormalTok{(firstRun)}
\KeywordTok{rm}\NormalTok{(t)}
\KeywordTok{rm}\NormalTok{(URL)}

\NormalTok{historicalPrices <-}\StringTok{ }\KeywordTok{subset}\NormalTok{(historicalPrices, date >=}\StringTok{ "2011/09/01"} \NormalTok{&}\StringTok{ }\NormalTok{date <=}\StringTok{ "2016/07/31"} \NormalTok{&}\StringTok{ }\NormalTok{!}\KeywordTok{is.na}\NormalTok{(ticker))}
\NormalTok{historicalPrices$ticker <-}\StringTok{ }\KeywordTok{as.character}\NormalTok{(historicalPrices$ticker)}
\NormalTok{historicalPrices$flag <-}\StringTok{ }\KeywordTok{as.character}\NormalTok{(historicalPrices$flag)}

\NormalTok{firstRun <-}\StringTok{ }\OtherTok{TRUE}
\NormalTok{for (i in }\DecValTok{2011}\NormalTok{:}\DecValTok{2016}\NormalTok{) \{}
  \NormalTok{for (j in }\DecValTok{1}\NormalTok{:}\DecValTok{4}\NormalTok{) \{}
    \NormalTok{if (!(i ==}\StringTok{ }\DecValTok{2016} \NormalTok{&}\StringTok{ }\NormalTok{j ==}\StringTok{ }\DecValTok{4}\NormalTok{)) \{}
      \NormalTok{URL <-}\StringTok{ }\KeywordTok{paste}\NormalTok{(}\StringTok{"ftp://ftp.sec.gov/edgar/full-index/"}\NormalTok{, i,}\StringTok{"/QTR"}\NormalTok{,j,}\StringTok{"/master.zip"}\NormalTok{,}\DataTypeTok{sep =} \StringTok{""}\NormalTok{)}
      \NormalTok{FNC <-}\StringTok{ }\KeywordTok{paste}\NormalTok{(}\StringTok{"master_QTR"}\NormalTok{, j, }\StringTok{"_"}\NormalTok{, i, }\StringTok{".zip"}\NormalTok{, }\DataTypeTok{sep =} \StringTok{""}\NormalTok{)}
      \NormalTok{FNU <-}\StringTok{ }\KeywordTok{paste}\NormalTok{(}\StringTok{"master_QTR"}\NormalTok{, j, }\StringTok{"_"}\NormalTok{, i, }\StringTok{".idx"}\NormalTok{, }\DataTypeTok{sep =} \StringTok{""}\NormalTok{)}
      \NormalTok{if (!}\KeywordTok{file.exists}\NormalTok{(FNC))}
        \KeywordTok{download.file}\NormalTok{(URL, FNC, }\DataTypeTok{method =} \StringTok{"libcurl"}\NormalTok{)}
      \KeywordTok{unzip}\NormalTok{(FNC, }\StringTok{"master.idx"}\NormalTok{)}
      \KeywordTok{file.rename}\NormalTok{(}\StringTok{"master.idx"}\NormalTok{, FNU)}
      \NormalTok{if (firstRun ==}\StringTok{ }\OtherTok{TRUE}\NormalTok{) \{}
        \NormalTok{filingDates <-}\StringTok{ }\KeywordTok{fread}\NormalTok{(FNU)}
        \NormalTok{filingDates <-}\StringTok{ }\KeywordTok{subset}\NormalTok{(filingDates, V3 ==}\StringTok{ "10-K"} \NormalTok{|}\StringTok{ }\NormalTok{V3 ==}\StringTok{ "10-Q"}\NormalTok{)}
        \NormalTok{firstRun <-}\StringTok{ }\OtherTok{FALSE}
      \NormalTok{\} else \{}
        \NormalTok{tmp <-}\StringTok{ }\KeywordTok{fread}\NormalTok{(FNU)}
        \NormalTok{tmp <-}\StringTok{ }\KeywordTok{subset}\NormalTok{(tmp, V3 ==}\StringTok{ "10-K"} \NormalTok{|}\StringTok{ }\NormalTok{V3 ==}\StringTok{ "10-Q"}\NormalTok{)}
        \NormalTok{filingDates <-}\StringTok{ }\KeywordTok{rbind}\NormalTok{(filingDates, tmp)}
      \NormalTok{\}}
    \NormalTok{\}}
  \NormalTok{\}}
\NormalTok{\}}

\KeywordTok{colnames}\NormalTok{(filingDates) <-}\StringTok{ }\KeywordTok{c}\NormalTok{(}\StringTok{"CIK"}\NormalTok{,}\StringTok{"company"}\NormalTok{,}\StringTok{"filing"}\NormalTok{,}\StringTok{"date"}\NormalTok{,}\StringTok{"location"}\NormalTok{)}
\KeywordTok{rm}\NormalTok{(i)}
\KeywordTok{rm}\NormalTok{(j)}
\KeywordTok{rm}\NormalTok{(URL)}
\KeywordTok{rm}\NormalTok{(firstRun)}
\KeywordTok{rm}\NormalTok{(tmp)}

\NormalTok{for (n in }\DecValTok{1}\NormalTok{:}\KeywordTok{nrow}\NormalTok{(stocks)) \{}
  \NormalTok{tmp <-}\StringTok{ }\KeywordTok{data.frame}\NormalTok{(}\KeywordTok{subset}\NormalTok{(filingDates, company %like%}\StringTok{ }\NormalTok{stocks$company[n]),stocks$ticker[n])}
  \NormalTok{if (!}\KeywordTok{exists}\NormalTok{(}\StringTok{"relevantFilings"}\NormalTok{))}
    \NormalTok{relevantFilings <-}\StringTok{ }\NormalTok{tmp}
  \NormalTok{else}
    \NormalTok{relevantFilings <-}\StringTok{ }\KeywordTok{rbind}\NormalTok{(relevantFilings, tmp)}
\NormalTok{\}}
\NormalTok{relevantFilings <-}\StringTok{ }\KeywordTok{data.frame}\NormalTok{(}\KeywordTok{as.Date}\NormalTok{(relevantFilings$date, }\StringTok{"%Y-%m-%d"}\NormalTok{), relevantFilings$stocks.ticker.n.,}\DecValTok{1}\NormalTok{)}
\KeywordTok{colnames}\NormalTok{(relevantFilings) <-}\StringTok{ }\KeywordTok{c}\NormalTok{(}\StringTok{"date"}\NormalTok{, }\StringTok{"ticker"}\NormalTok{, }\StringTok{"flag"}\NormalTok{)}
\NormalTok{PriceChanges <-}\StringTok{ }\KeywordTok{data.frame}\NormalTok{(}\KeywordTok{as.Date}\NormalTok{(historicalPrices$date,}\StringTok{"%Y-%m-%d"}\NormalTok{), historicalPrices$ticker, historicalPrices$price)}
\KeywordTok{colnames}\NormalTok{(PriceChanges) <-}\StringTok{ }\KeywordTok{c}\NormalTok{(}\StringTok{"date"}\NormalTok{, }\StringTok{"ticker"}\NormalTok{, }\StringTok{"price"}\NormalTok{)}
\end{Highlighting}
\end{Shaded}

\emph{Note} BUG: This block will hang the script when SEC.GOV ftp site
is down. An alternative is to have the required files already in the
working directory as it will make the script skip the download portion.

The following stores the options price for calls and puts on the
selected date (\emph{note}: this step is necessary due to not having the
data for all dates and other stocks. The code below can easily be
expanded to include the entire dataset if avaible).

\begin{Shaded}
\begin{Highlighting}[]
\NormalTok{fdate <-}\StringTok{ }\KeywordTok{as.Date}\NormalTok{(}\StringTok{"2013-04-29"}\NormalTok{, }\StringTok{"%Y-%m-%d"}\NormalTok{)}
\NormalTok{fdprev <-}\StringTok{ }\KeywordTok{as.Date}\NormalTok{(}\StringTok{"2013-04-26"}\NormalTok{, }\StringTok{"%Y-%m-%d"}\NormalTok{)}
\NormalTok{fdnext <-}\StringTok{ }\KeywordTok{as.Date}\NormalTok{(}\StringTok{"2013-05-02"}\NormalTok{, }\StringTok{"%Y-%m-%d"}\NormalTok{)}
\NormalTok{price <-}\StringTok{ }\NormalTok{PriceChanges$price[PriceChanges$date ==}\StringTok{ }\NormalTok{fdate]}
\NormalTok{prv <-}\StringTok{ }\KeywordTok{fread}\NormalTok{(}\StringTok{"UnderlyingOptionsEODQuotes_2013-04-26.csv"}\NormalTok{, }\DataTypeTok{select =} \KeywordTok{c}\NormalTok{(}\StringTok{"strike"}\NormalTok{, }\StringTok{"expiration"}\NormalTok{, }\StringTok{"option_type"}\NormalTok{, }\StringTok{"bid_1545"}\NormalTok{, }\StringTok{"ask_1545"}\NormalTok{))}
\NormalTok{nxt <-}\StringTok{ }\KeywordTok{fread}\NormalTok{(}\StringTok{"UnderlyingOptionsEODQuotes_2013-05-02.csv"}\NormalTok{, }\DataTypeTok{select =} \KeywordTok{c}\NormalTok{(}\StringTok{"strike"}\NormalTok{, }\StringTok{"expiration"}\NormalTok{, }\StringTok{"option_type"}\NormalTok{, }\StringTok{"bid_1545"}\NormalTok{, }\StringTok{"ask_1545"}\NormalTok{))}
\end{Highlighting}
\end{Shaded}

With the correct range selected, we can now create a straddle and
measure it's returns.

\begin{Shaded}
\begin{Highlighting}[]
\NormalTok{strikeList <-}\StringTok{ }\KeywordTok{sort}\NormalTok{(}\KeywordTok{unique}\NormalTok{(prv$strike))}
\NormalTok{lp <-}\StringTok{ }\KeywordTok{data.frame}\NormalTok{(strikeList, }\KeywordTok{abs}\NormalTok{(price -}\StringTok{ }\NormalTok{strikeList))}
\KeywordTok{colnames}\NormalTok{(lp) <-}\StringTok{ }\KeywordTok{c}\NormalTok{(}\StringTok{"strike"}\NormalTok{,}\StringTok{"dist"}\NormalTok{)}
\NormalTok{strikePrice <-}\StringTok{ }\NormalTok{lp$strike[lp$dist ==}\StringTok{ }\KeywordTok{min}\NormalTok{(lp$dist)]}

\CommentTok{# Cost is the sum of a call and a put in the prv variable, which is days before the filing releases}
\NormalTok{cost <-}\StringTok{ }\KeywordTok{subset}\NormalTok{(prv, strike ==}\StringTok{ }\NormalTok{strikePrice &}\StringTok{ }\NormalTok{expiration ==}\StringTok{ "2013-05-03"} \NormalTok{&}\StringTok{ }\NormalTok{option_type ==}\StringTok{ "C"}\NormalTok{)$ask_1545 +}\StringTok{ }\KeywordTok{subset}\NormalTok{(prv, strike ==}\StringTok{ }\NormalTok{strikePrice &}\StringTok{ }\NormalTok{expiration ==}\StringTok{ "2013-05-03"} \NormalTok{&}\StringTok{ }\NormalTok{option_type ==}\StringTok{ "P"}\NormalTok{)$ask_1545}

\NormalTok{rev <-}\StringTok{ }\KeywordTok{subset}\NormalTok{(nxt, strike ==}\StringTok{ }\NormalTok{strikePrice &}\StringTok{ }\NormalTok{expiration ==}\StringTok{ "2013-05-03"} \NormalTok{&}\StringTok{ }\NormalTok{option_type ==}\StringTok{ "C"}\NormalTok{)$bid_1545 +}\StringTok{ }\KeywordTok{subset}\NormalTok{(nxt, strike ==}\StringTok{ }\NormalTok{strikePrice &}\StringTok{ }\NormalTok{expiration ==}\StringTok{ "2013-05-03"} \NormalTok{&}\StringTok{ }\NormalTok{option_type ==}\StringTok{ "P"}\NormalTok{)$bid_1545}

\NormalTok{rev -}\StringTok{ }\NormalTok{cost}
\end{Highlighting}
\end{Shaded}

\begin{verbatim}
## [1] 7.3
\end{verbatim}

Now, let's try do this for all strike prices and all expiration dates
(greater than the current date)

\begin{Shaded}
\begin{Highlighting}[]
\NormalTok{firstTime <-}\StringTok{ }\OtherTok{TRUE}

\NormalTok{for (s in }\KeywordTok{unique}\NormalTok{(nxt$strike)) \{}
  \NormalTok{for (exp in }\KeywordTok{unique}\NormalTok{(nxt$expiration)) \{}
    \NormalTok{cost <-}\StringTok{ }\KeywordTok{subset}\NormalTok{(prv, strike ==}\StringTok{ }\NormalTok{s &}\StringTok{ }\NormalTok{expiration ==}\StringTok{ }\NormalTok{exp &}\StringTok{ }\NormalTok{option_type ==}\StringTok{ "C"}\NormalTok{)$ask_1545 +}\StringTok{ }\KeywordTok{subset}\NormalTok{(prv, strike ==}\StringTok{ }\NormalTok{s &}\StringTok{ }\NormalTok{expiration ==}\StringTok{ }\NormalTok{exp &}\StringTok{ }\NormalTok{option_type ==}\StringTok{ "P"}\NormalTok{)$ask_1545}
    
    \NormalTok{rev <-}\StringTok{ }\KeywordTok{subset}\NormalTok{(nxt, strike ==}\StringTok{ }\NormalTok{s &}\StringTok{ }\NormalTok{expiration ==}\StringTok{ }\NormalTok{exp &}\StringTok{ }\NormalTok{option_type ==}\StringTok{ "C"}\NormalTok{)$bid_1545 +}\StringTok{ }\KeywordTok{subset}\NormalTok{(nxt, strike ==}\StringTok{ }\NormalTok{s &}\StringTok{ }\NormalTok{expiration ==}\StringTok{ }\NormalTok{exp &}\StringTok{ }\NormalTok{option_type ==}\StringTok{ "P"}\NormalTok{)$bid_1545}
    
    \NormalTok{if (}\KeywordTok{length}\NormalTok{((rev -}\StringTok{ }\NormalTok{cost)) ==}\StringTok{ }\DecValTok{0}\NormalTok{)}
      \NormalTok{profit =}\StringTok{ }\DecValTok{0}
    \NormalTok{else}
      \NormalTok{profit =}\StringTok{ }\NormalTok{rev -}\StringTok{ }\NormalTok{cost}
    
    \NormalTok{if (firstTime ==}\StringTok{ }\OtherTok{TRUE}\NormalTok{) \{}
      \NormalTok{firstTime <-}\StringTok{ }\OtherTok{FALSE}
      \NormalTok{results <-}\StringTok{ }\KeywordTok{data.frame}\NormalTok{(exp,s,profit)}
    \NormalTok{\} else \{}
      \NormalTok{tmp <-}\StringTok{ }\KeywordTok{data.frame}\NormalTok{(exp,s,profit)}
      \NormalTok{results <-}\StringTok{ }\KeywordTok{rbind}\NormalTok{(results,tmp)}
    \NormalTok{\}}
  \NormalTok{\}}
\NormalTok{\}}
\KeywordTok{colnames}\NormalTok{(results) <-}\StringTok{ }\KeywordTok{c}\NormalTok{(}\StringTok{"expiration_date"}\NormalTok{,}\StringTok{"strike_price"}\NormalTok{,}\StringTok{"profit"}\NormalTok{)}
\end{Highlighting}
\end{Shaded}

\section{Results}\label{results}

In the above block we got all the results in the results vector. To
avoid making this document huge we are skipping plotting the table and
most of the results can be seen by the end of the document when subsets
of the data are plotted (3-variable table with negative some negative
values in profits, this bubble charts can't be used - solution was to
plot subsets of the data).

In summary, using the data sample we can see that the profitability
increases when the strike prices are very close to the current
underlying stock price and with options with strike price close to or
below the current stock price. Unfortunately, this can't prove or
disprove any strategy due to the lack of freely available data.
Discussion will comment more on this.

\section{Dicussion}\label{dicussion}

The initial goal of this paper was to measure the returns on a
systematic stock options strategy using straddles. However, one
unexpected problem showed in the middle stages of the paper: the
non-availability of public historical options price data. For this
reason, this paper had to be reduced in scope, to the test of the just
one sample available. Even though this sample shows promising returns in
the sample, it is by no means proof that it would work for other stocks.

Assuming that this data becomes available in the future, the scripts
written in this paper can easily be extended to include a large dataset.
Other tests also becomes feasible, like matching black-scholles pricing
to the current pricing, seasonaility control, outlier removal and so on.

\newpage

\section{References}\label{references}

\begin{figure}[htbp]
\centering
\includegraphics{CorrocherFilho_Nelson_Project_Proposal_files/figure-latex/unnamed-chunk-2-1.pdf}
\caption{\label{fig:unnamed-chunk-2}Plotting with fixed expiration date of
2013-05-03}
\end{figure}

\begin{figure}[htbp]
\centering
\includegraphics{CorrocherFilho_Nelson_Project_Proposal_files/figure-latex/unnamed-chunk-3-1.pdf}
\caption{\label{fig:unnamed-chunk-3}Plotting with fixed expiration date of
2013-06-22}
\end{figure}

\begin{figure}[htbp]
\centering
\includegraphics{CorrocherFilho_Nelson_Project_Proposal_files/figure-latex/unnamed-chunk-4-1.pdf}
\caption{\label{fig:unnamed-chunk-4}Plotting with fixed expiration date of
2013-07-20}
\end{figure}

\begin{figure}[htbp]
\centering
\includegraphics{CorrocherFilho_Nelson_Project_Proposal_files/figure-latex/unnamed-chunk-5-1.pdf}
\caption{\label{fig:unnamed-chunk-5}Plotting with fixed strike price at 160}
\end{figure}

\begin{figure}[htbp]
\centering
\includegraphics{CorrocherFilho_Nelson_Project_Proposal_files/figure-latex/unnamed-chunk-6-1.pdf}
\caption{\label{fig:unnamed-chunk-6}Plotting with fixed strike price at 190}
\end{figure}

\begin{figure}[htbp]
\centering
\includegraphics{CorrocherFilho_Nelson_Project_Proposal_files/figure-latex/unnamed-chunk-7-1.pdf}
\caption{\label{fig:unnamed-chunk-7}Plotting with fixed strike price at 210}
\end{figure}

\section{Session Info}\label{session-info}

\begin{Shaded}
\begin{Highlighting}[]
\KeywordTok{sessionInfo}\NormalTok{()}
\end{Highlighting}
\end{Shaded}

\begin{verbatim}
## R version 3.3.2 (2016-10-31)
## Platform: x86_64-w64-mingw32/x64 (64-bit)
## Running under: Windows 10 x64 (build 10586)
## 
## locale:
## [1] LC_COLLATE=English_United States.1252 
## [2] LC_CTYPE=English_United States.1252   
## [3] LC_MONETARY=English_United States.1252
## [4] LC_NUMERIC=C                          
## [5] LC_TIME=English_United States.1252    
## 
## attached base packages:
## [1] stats     graphics  grDevices utils     datasets  methods   base     
## 
## other attached packages:
## [1] ggplot2_2.2.0     xlsx_0.5.7        xlsxjars_0.6.1    rJava_0.9-8      
## [5] data.table_1.10.0 papaja_0.1.0.9456
## 
## loaded via a namespace (and not attached):
##  [1] Rcpp_0.12.8      knitr_1.15.1     magrittr_1.5     munsell_0.4.3   
##  [5] colorspace_1.3-1 highr_0.6        stringr_1.1.0    plyr_1.8.4      
##  [9] tools_3.3.2      grid_3.3.2       gtable_0.2.0     htmltools_0.3.5 
## [13] assertthat_0.1   yaml_2.1.14      lazyeval_0.2.0   rprojroot_1.1   
## [17] digest_0.6.10    tibble_1.2       bookdown_0.3     evaluate_0.10   
## [21] rmarkdown_1.2    stringi_1.1.2    scales_0.4.1     backports_1.0.4
\end{verbatim}

\setlength{\parindent}{-0.5in} \setlength{\leftskip}{0.5in}

\hypertarget{refs}{}
\hypertarget{ref-R-papaja}{}
Aust, F., \& Barth, M. (2016). \emph{Papaja: Create apa manuscripts with
rmarkdown}. Retrieved from \url{https://github.com/crsh/papaja}

\hypertarget{ref-berkdemarzo}{}
Berk, P. M., Jonathan BDeMarzo. (n.d.). \emph{Corporate finance}.

\hypertarget{ref-bodie}{}
Bodie, Z., Kane, A., \& Marcus, A. J. (n.d.). \emph{Investments}.

\hypertarget{ref-R-data.table}{}
Dowle, M., Srinivasan, A., Short, T., R Saporta, S. L. with
contributions from, \& Antonyan, E. (2015). \emph{Data.table: Extension
of data.frame}. Retrieved from
\url{https://CRAN.R-project.org/package=data.table}

\hypertarget{ref-R-xlsx}{}
Dragulescu, A. A. (2014a). \emph{Xlsx: Read, write, format excel 2007
and excel 97/2000/xp/2003 files}. Retrieved from
\url{https://CRAN.R-project.org/package=xlsx}

\hypertarget{ref-R-xlsxjars}{}
Dragulescu, A. A. (2014b). \emph{Xlsxjars: Package required poi jars for
the xlsx package}. Retrieved from
\url{https://CRAN.R-project.org/package=xlsxjars}

\hypertarget{ref-straddle}{}
Investopedia. (2003). \emph{Straddle definition}. Retrieved from
\url{http://www.investopedia.com/terms/s/straddle.asp}

\hypertarget{ref-CBOE}{}
\emph{Options, equity \& etf - tick \& trade data -- cboe livevol data
shop}. (2017). Retrieved from \url{https://datashop.cboe.com/}

\hypertarget{ref-R-base}{}
R Core Team. (2016). \emph{R: A language and environment for statistical
computing}. Vienna, Austria: R Foundation for Statistical Computing.
Retrieved from \url{https://www.R-project.org/}

\hypertarget{ref-sec_1}{}
Sec.gov \textbar{} company search page\_2016. (2016). \emph{Sec.gov}.
Retrieved from
\url{https://www.sec.gov/edgar/searchedgar/companysearch.html}

\hypertarget{ref-sec_2}{}
Sec.org \textbar{} information for ftp users - 2016. (2016).
\emph{Sec.gov}. Retrieved from
\url{https://www.sec.gov/edgar/searchedgar/ftpusers.htm}

\hypertarget{ref-R-rJava}{}
Urbanek, S. (2016). \emph{RJava: Low-level r to java interface}.
Retrieved from \url{https://CRAN.R-project.org/package=rJava}

\hypertarget{ref-R-ggplot2}{}
Wickham, H. (2009). \emph{Ggplot2: Elegant graphics for data analysis}.
Springer-Verlag New York. Retrieved from \url{http://ggplot2.org}

\hypertarget{ref-yahoo_1}{}
Yahoo finance - business finance, stock market, quotes, news\_2016.
(2016). \emph{Finance.yahoo.com}. Retrieved from
\url{http://finance.yahoo.com/}






\end{document}
